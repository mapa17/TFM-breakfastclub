\chapter{Introduction}
This document is describing the Master Thesis developed by Manuel Pasieka as part
of the Master in Artificial Intelligence at UNIVERSIDAD INTERNACIONAL DE LA RIOJA, S.A.
2018-2019. \par

\bb

As part of the Thesis the student developed an Agent-based simulation named
\textbf{Breakfastclub} (available at http://github.com/mapa17/breakfastclub) of a virtual classroom in order to study the effect of
different Personality Traits on happiness and attention in the simulated class.
This document is describing the development of the project and the results achieved.

\bb

The document is split into the following chapters.
\begin{itemize}
\item This first chapter introduces the reader into the motivation behind this work and
its novelties.
\item The second chapter will discuss the state of the art of the methods and technologies
applied.
\item The third chapter lays out the initials objectives as well as their adaption
and final objectives of the Thesis.
\item The fourth chapter describes in detail the implementation and technical solution
to the proposed problem.
\item The fifth chapter is focused on the Data Analysis of the results generated.
\item The sixth chapter is providing a conclusion and summary of what has been presented.
\end{itemize}

\section{Origin and Motivation}
Human beings are social animals, and although this has been acknowledged by
in psychology and the social studies for a long time [NOTE: References?], it is 
our claim that there is a lack of adequate tools and techniques to study in detail
the behavior of humans in many different group settings.

In particular we are interested in the group dynamics of children in a classroom,
engaged in a autonomous study group.

\bb

We have two goals for this work
\begin{enumerate}
    \item Develop a flexible and extendable multi agent based simulation of a virtual classroom
    \item Study how different personality traits effect attention and happiness of individuals and the group as a whole
\end{enumerate}
