\chapter{Objectives}
As it is typical for projects beyond certain size, the objectives and scope had to
be adapted according to the progression of the project. This chapter describes the initial
envisioned objectives as well as the objectives reached with the final version
developed during as part of the thesis.

One of the initial objectives of the thesis was to develop a simulation environment that
could be used interactively as well as a closed loop simulation (i.e. once defined
and setup would run without any user interaction until a defined state is reached).

In addition as the simulation is based on Unity3d, the Machine Learning Package
would have be used to implement agents trained using a Reinforcement Learning approach.

Because of time contains the interaction and ML-Agent features have not been included
in the final version developed during the Thesis, but are described in the last chapter
on the outlook of the project.

\bb

The objectives for the final version based on the resources and and time available
are therefor the following:

\begin{itemize}
    \item \textbf{Closed Loop Simulation:} Implementation of Unity3d based virtual
    classroom simulation, including a 2D top down visualization.
    \item \textbf{Psychological agent model:} Development of a psychological model
    governing the behavior of agents that is based on empirical and theoretical grounds.
    \item \textbf{Seedable and deterministic:} The closed loop simulation should be
    deterministic and the random components should be seedable, making it possible
    to reproduce results of previous simulations if the same seed is provided.
    \item \textbf{Simulation and Classroom configuration:} The simulation as well
    as well as the psychological profile of the classroom should be easily configurable
    and alterable without the need to modify the simulation software.
    \item \textbf{Agent and Classroom based analysis:} As part of the data analysis
    it should be possible to analyze the behavior of individual agents (i.e students)
    as well as the average of the complete classroom (i.e. group).
    \item \textbf{Comparison of pre-defined psychological classroom profiles:} Based
    on empirical pedagogical studies a defined set of psychological interesting classroom
    profiles are compared to each other.
\end{itemize}

