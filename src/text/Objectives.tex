\chapter{Objectives and Methodology}
The main objective of Thesis is the development of a simulation using an
agent-based model that is capable of simulation students in an virtual classroom.

\bb

As it is typical for projects beyond certain size, the objectives and scope had to
be adapted according to the progression of the project after the initial planning phase.
This chapter describes the initial envisioned objectives as well as the objectives
reached with the final version of the thesis.

One of the initial objectives of the thesis was to develop a simulation environment that
could be used interactively as well as a closed loop simulation (i.e. once defined
and setup would run without any user interaction until a defined state is reached).

In addition as the simulation is based on Unity3d, the Machine Learning Package
was intended to be used to implement agents trained using a Reinforcement Learning approach.

Because of time constrains the interaction and ML-Agent features have not been included
in the final version developed during the Thesis, but are included in the last chapter
on the outlook of the project.

\section{Particular Objectives}

The objectives for the final version based on the resources and time available
have been the following:

\begin{itemize}
    \item \textbf{Closed Loop Simulation:} Implementation of Unity3d based virtual
    classroom simulation, including a 2D top down visualization.
    \item \textbf{Psychological agent model:} Development of a psychological model
    governing the behavior of agents that is based on empirical and theoretical grounds.
    \item \textbf{Deterministic Simulation:} The closed loop simulation should be
    deterministic and the random components should be seedable, making it possible
    to reproduce results of previous simulations if the same seed is provided.
    \item \textbf{Simulation and Classroom configuration:} The simulation as well
    as well as the psychological profile of the classroom should be easily configurable
    and alterable without the need to modify the simulation software.
    \item \textbf{Agent and Classroom based analysis:} As part of the data analysis
    it should be possible to analyze the behavior of individual agents (i.e students)
    as well as the average of the complete classroom (i.e. group).
    \item \textbf{Comparison of pre-defined psychological classroom profiles:} Based
    on empirical pedagogical studies a defined set of psychological interesting classroom
    profiles are compared to each other.
\end{itemize}

\section{Methodology}
The development of the simulation was following an \textit{agile software development}
principle in which based on two prototypes, and many small iterations the final
software was developed.

\bb

In order to keep track of the project progress and open tasks, the issue and project
management system that is part of Github was used. There we defined two main milestones
for the final simulation, describing goals and associated tasks.
During the development progress, any arising change was documented and the associated
code changes have been managed as Github issues.

\bb

Concerning the agent mechanics and simulation parameters, a iterative process was
pursued. The behavior of agents was observed manually using the interactive simulation
for any abnormal or undesired behavior. Once an issue was identified, the agent logic
was adapted to resolve the issue, and its correctness verified observing a the
simulation process.

\bb

Simulation Parameters have been chosen using the data analysis scripts developed
as part of the thesis (see chapter \ref{Chapter:DataAnalysis}). The behavior
of individual agents, as well as classroom aggregates have been used to find
the simulation parameters used to perform the study (Simulation parameters are included
in the Appendix). The parameters have been tuned using three classroom profiles,
following different objectives.

\begin{itemize}
    \item \textbf{Normal classroom:} A classroom with 30 students having a 'normal'
    personality profile, have been used to define thresholds like noise, and happiness
    and motivation increases.
    \item \textbf{Test classroom:} A special test class containing different type of
    personality profiles has been used to verify that simulated agent behavior is
    corresponding to results gathered in empirical studies.
    \item \textbf{Random classroom:} A special classroom with students having random
    personality profiles was used to maximize the difference in happiness and attention
    between different agents.
\end{itemize}
