\chapter{Abstract}

Agent-based models have proven to be a useful tool to study complex social phenomena.
In this work we have developed a simulation using an agent-based model of a virtual
classroom, simulating the behavior of students and adolescents in an autonomous
study group. The agent cognition is based on the widely used Big-Five personality
trait model, and agent behavior has been aligned with empirical studies showing
how specific personality traits correlate with academic success. The simulation
software was used to compare how different classroom compositions with an increasing
ratio of students with Attention-deficit hyperactivity disorder (ADHD) prototypical
personality traits affect the classroom dynamics. We found a very strong effect
of ADHD students on the mean classroom happiness and attention. Even a very small
number of ADHD students can cause a shift in the behavior of None-ADHD students,
decreasing their mean happiness and attention, in addition to more frequent classroom
wide quarrels.

\bb

The simulation software in addition with the data analysis pipeline is available
open source and under the MIT license. 

\bb

{\bf Keywords:} Agent-based model, Big Five, classroom, ADHD

\pagebreak


Model multi-agente son herramientas útiles para estudiar fenómenos sociales complejos.
Como parte de esta tesis hemos desarrollado una simulación multi-agente para
simular una clase virtual de un grupo de estudiantes autónomas. El modelo
cognitivo de los agentes se base en el modelo de personalidad de los cinco grandes
(BigFive), y genera un comportamiento que esta replicando resultados empíricos.
Para demonstrar las capacidades de la simulación hemos simulado como se cambia
la atención y felicidad media de una clase en función del número de estudiantes
con personalidades típicos de niños con trastorno por déficit de atención con
hiperactividad (TDAH). Hemos observado un efecto fuerte de los alumnos con TDAH
a la clase entera. Hasta un número muy bajo de estos alumnos produce una
reducción significante de atención y felicidad media, y además aumenta las
disputas en la clase.

\bb

La simulación y las herramientas de análisis de datos están disponible como open
source bajo de la licencia MIT.

\bb

{\bf Palabras Clave:} modelo multi-agente, Big Five, aula de clase, TDAH
