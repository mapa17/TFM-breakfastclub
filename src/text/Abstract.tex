\chapter{Abstract}

Agent-based models have proven to be a useful tool to study complex social phenomena.
In this work we have developed a simulation using an agent-based model of a virtual
classroom, simulating the behavior of children and adolescents in an autonomous
study group. The agent cognition is based on the widely used Big-Five personality
trait model, and agent behavior has been aligned with empirical studies showing
how specific personality traits correlate with academic success. The simulation
software was used to compare how different classroom compositions with an increasing
ratio of children with Attention-deficit hyperactivity disorder (ADHD) typical
personality traits affect the classroom dynamics.

\bb

The simulation software in addition with the data analysis pipeline is available
open source and under the MIT license. 

\bb

{\bf Keywords:} Agent-based model, Big Five, classroom, ADHD