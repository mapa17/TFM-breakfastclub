\label{StateOfTheArt}
\chapter{State of the art}
In this chapter we will provide an overview of existing simulations software, and
software particularly developed for virtual classrooms. We describe the deficiencies
of those solutions, and why we believe that there is a necessity to develop our
own simulation.

\section{Social Simulations and Agent-based models}
Social Simulations are a special type of Agent-based models (\textbf{ABM}) that are used to study
the dynamics of groups based on the individual agents that respond to their own
and other agents expectations\cite{Helbing2012}.

NOTE[Have some definition + explanation of what is an agent-based model]

\bb

Social Simulations have been used since the early 1960s to study the dynamics of
social segregation\cite{Schelling1971} or more recently on the spread of
contagious diseases\cite{Perez2009} within a city.

The intention of many of those simulations is to find emerging properties of
the complete system (i.e. the group), which are absent in the individual agents\cite{Jackson2017}.
This is of particular interest if the the found properties are empirically verified,
but not understood. 

Simulated systems have the great benefit to be easy to manipulate and produce almost
no costs to run, making them an excellent tool to study complex dynamic systems.

\section{Agent-based model Software}
A series of open source and commercial distributed Agent Software\cite{Kravi2015}
is available. Some of the most popular ones are NetLogo\cite{Tissue2004},
Swarm [NOTE: Reference for Swarm], or Mesa\cite{Masad2015}, which provide a framework
to develop multi-agent simulations, often including a simple Visualization and
a GUI.

\bb

We decided against against using those existing frameworks, and instead develop
our own solution based on the Unity3d [Note: REF???] Game and Simulation
Engine.

In particular Unity3d provides us with the following features that are absent
or underdeveloped in the other frameworks.

\begin{itemize}
    \item \textbf{State of the Art Visualization:} Unity3D is used to develop triple
    A computer games and provides the possibility to build simulations with realistic
    appearing Visuals and even virtual reality environments.
    \item \textbf{User Interaction:} User interaction if present at all is implemented
    very poorly in the other simulation frameworks. As User Interaction is essential
    part in every computer game, Unity3d provides an excellent support for it.
    \item \textbf{Integration with Machine Learning tools:} In the last year
    Unity3d has been extending its capabilities as a Agent based modeling framework
    by including a Machine Learning Agent toolkit that provides easy
    interface between State of the Art machine Learning Tools like Tensorflow or Pytorch
    and the Unity simulation.
    \item \textbf{Actively Maintained:} Many simulation frameworks have been academic endeavors
    with a short lifespan, and on multiple occasions stopped to be maintained and
    to be usable after a short period of time. Relying on a commercial sustained
    framework like Unity3d ensures availability and eases future development of
    the project.
\end{itemize}

Those aspects are not essential to the present solution but provide use with
the possibility to extend the system as is discussed in the Outlook, in the final
chapter of the Thesis.

\section{Simulations of virtual Classrooms}
Of particular interest to us are Simulation Systems that focus on a virtual classroom. 
Several academic and commercial systems have been developed with different objectives in mind.

\bb

Some of those solutions (e.g. TLE TeachLivE\cite{Dieker2014}\cite{Dieker2017} or
simSchool \cite{Badiee2015}) focus on teacher education, providing a virtual classroom
that can be used for new teachers to learn how to interact with a class and resolve issues.
Others (e.g. Katana Sim:Classroom \cite{Blume2019}) are used as a simulation
environment for academic research, focusing on psychological studies.

\bb

Evaluating the different simulations we found that all of them lacked one or more
of the following features, and therefore decided to develop our own solution.

\begin{itemize}
    \item \textbf{OpenSource:} The Simulation should be open source and freely available
    for academic and commercial purposes, in order to support its adoption and support
    the sustainability of the project.
    \item \textbf{Model of Agent Logic:} The agent behavior should depend on an
    flexible agent logic that is based on empirical psychological studies.
    \item \textbf{Flexibility:} The simulation should be configure able to scale
    class size, student profiles and classroom environment.
    \item \textbf{Reproducibility:} The simulation outcome (except of user interaction) should
    be reproducible, in order to provide a framework to study particular group dynamics.
    If multiple runs of the same simulation produce different results, it is unclear
    how alterations of the simulation configuration effect the outcome.
    \item \textbf{Data Analysis:} The simulation should include methods and tools
    to study its results. In particular it should be possible to execute multiple
    instances of the simulation with slightly changed configurations in order to
    perform a statistical analysis of the outcome.
\end{itemize}
