\label{StateOfTheArt}
\chapter{State of the art}
In this chapter we will provide introduction into what agent-based models are and
how they are used in the context of social studies. We will provide an overview
of existing simulations platforms, and software particularly developed to simulate
a virtual classrooms. We discuss the deficiencies of those systems, and why
we decided to develop our own.

\section{Social Simulations and Agent-based models}
Agent-based models (\textbf{ABM})\cite{Jackson2017} have been used in various fields
to study complex systems that result from the interaction of many individual agents.

Examples of such systems are the stock market, crowds, beehives, social networks and many
more. Social Simulations are a type of ABM that focus on modeling social dynamics
of humans or animals \cite{Helbing2012}.

Early examples of such social simulations is Shellings work \cite{Schelling1971}
studding the dynamics of social segregation or more recently on the spread of
contagious diseases\cite{Perez2009} within a city.

\bb

The intention of many of those simulations is to find emerging social phenomena
present and empirically observed in the complete system (i.e. the group), but
which are absent in the individual agents\cite{Jackson2017}.
Hence is is the search for emerging properties, often observed but not understood,
that can be explained by the interaction of simple mechanisms of the individual agents.

\bb

In simulating such multi-agent systems one has to possibility to construct, monitor
and manipulate the system with perfect granularity and with very little cost, making
such simulations an excellent tool to study complex dynamic systems.

\section{Agent-based model Software}
A series of open source as well as commercial distributed Agent Software\cite{Kravi2015}
exists. Some of the more popular ones are NetLogo\cite{Tissue2004},
Swarm\cite{Minar1996}, or Mesa\cite{Masad2015}, that provide frameworks
to develop multi-agent simulations, often including visualization and
a GUI.

\bb

We decided against against using those existing frameworks, and instead develop
our own solution based on the Unity3d\footnote{Developed by Unity Technologies and available from https://unity.com}
Game and Simulation Engine.

In particular Unity3d provides us with the following features that are absent
or underdeveloped in other frameworks.

\begin{itemize}
    \item \textbf{State of the Art Visualization:} Unity3D is used to develop triple
    A computer games and provides the possibility to build simulations with realistic
    appearing visuals and even virtual reality environments.
    \item \textbf{User Interaction:} User interaction if present at all is implemented
    very poorly in the most simulation frameworks. As User Interaction is an essential
    part in every computer game, Unity3d provides an excellent support for that.
    \item \textbf{Integration with other Machine Learning tools:} In the last year
    Unity3d has been extending its capabilities as a Agent based modeling framework
    by including a Machine Learning Agent toolkit that provides an easy
    interface between State of the Art machine Learning Tools like Tensorflow or Pytorch
    and the Unity simulation.
    \item \textbf{Actively Maintained:} Many simulation frameworks have been academic endeavors
    with a short lifespan, and on multiple occasions stopped to be maintained and
    to be available after a short period of time. Relying on a commercial sustained
    framework like Unity3d ensures availability and eases future development of
    the project.
\end{itemize}

Although the current version of the simulation is not making full use of all those
features at the moment, Unity3d has been chosen to serve as a platform for future
development based on the results achieved during the thesis.

\section{Simulations of virtual Classrooms}
Of particular interest to us are Simulation Systems that focus on a virtual classroom. 
Several academic and commercial systems have been developed with different objectives.

\bb

Some of those solutions (e.g. TLE TeachLivE\cite{Dieker2014}\cite{Dieker2017} or
simSchool \cite{Badiee2015}) focus on teacher education, providing a virtual classroom
that can be used for new teachers to learn how to interact with a class and resolve issues.
Others (e.g. Katana Sim:Classroom \cite{Blume2019}) are used as a simulation
environment for academic research, focusing on psychological studies.

\bb

Evaluating the different simulations we found that all of them lacked one or more
of the following features, and therefore decided to develop our own solution.

\begin{itemize}
    \item \textbf{OpenSource:} The Simulation should be open source and freely available
    for academic and commercial purposes, in order to support its adoption and support
    the sustainability of the project.
    \item \textbf{Agent Model:} The agent behavior should depend on an
    flexible agent logic that is based on empirical psychological studies.
    \item \textbf{Flexibility:} The simulation should be configure able to scale
    class size, student profiles and classroom environment.
    \item \textbf{Reproducibility:} The simulation outcome (except of user interaction) should
    be reproducible, in order to provide a framework to study particular group dynamics.
    If multiple runs of the same simulation produce different results, it is unclear
    how alterations of the simulation configuration effect the outcome.
    \item \textbf{Data Analysis:} The simulation should include methods and tools
    to study the results generated. In particular it should be possible to execute multiple
    instances of the simulation with slightly changed conditions in order to
    perform a statistical analysis of the outcome.
\end{itemize}
