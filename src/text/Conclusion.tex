\chapter{Conclusion and outlook}
In this chapter we will draw our final conclusions, including an outlook on
possible directions to continue and improve the work done so far.

\section{Conclusion}
As part of the thesis we have developed a multi-agent model that is simulating a virtual classroom.

We have devised a Agent Logic that is based on the established Big-Five Personality
Trait model and that produces agent behavior consistent with empirical studies.

We have developed a data analysis pipeline that makes it possible to efficiently
run multiple simulations in a batch mode, enabling a statistical analysis of the
simulation results.

We chose to compare how students with prototypical ADHD personality traits affect
the classroom dynamics, and therefore defined multiple classrooms varying in 
ratio between none to a high number of ADHD students.

We found that students have a very strong effect on the average happiness and attention
of the class. It appears that ADHD students affect the behavior of None-ADHD Students
without changing their won. In addition we found that even a low number of
ADHD students cause more frequent riots, specially in classrooms with very
ambitious students.

The simulation software, the data analysis scripts and all content presented in this
thesis is available freely under the MIT License from the github repository
\href{https://github.com/mapa17/breakfastclub}{https://github.com/mapa17/breakfastclub}.



\section{Outlook}
As mentioned shortly in the chapter on Objectives some initial objectives had to
be dropped during the development of the thesis in order to stay within the available
time frame.

The following is a list of possible improvements, as well as ideas for follow up projects.

\begin{itemize}
    \item \textbf{Improve classroom analysis:}
    The HA-Plot for different classroom configurations is a very concise representation
    of multiple instances of the same classroom configuration. Many of the temporal
    dynamics shown in the classroom aggregate plot are not visible in the HA-Plot.
    In addition there is no direct way to compare classroom aggregate plots.

    \bb

    Methods from Time series analysis could be used to extract information about
    the signal dynamics (like periodicity, entropy, moments, ...) in order to compare
    instances of the same classroom configuration or between different configurations.

    \bb

    It would be particularly useful extract a \textit{typical} classroom aggregate
    showing the classroom dynamics for an configuration and not only an instance.
    \item \textbf{Interactive Simulation:} 
    On way to extend how the simulation can be used, is to make the simulation
    interactive. Doing so would provide the user with the option to force agents
    to perform certain actions, expulse students from the classroom, call for silence
    or similar interactions. This would make it possible to develop a teacher training 
    program, based on the simulator, similar to commercial solutions like TLE TeachLive
    \cite{Dieker2017} or simSchool \cite{Badiee2015} but open source and with
    agent behavior that is based on a psychological model.
    
    \bb
    
    In order to support learning and provide a novel visualization of the effect
    of user interaction with the simulation we envisioned a system that is able to track
    the effect of each user interaction onto the final result of the simulation.
    This could be achieved by creating a clone of the running simulation when ever
    the user is performing an action. That clone instance would continue until the 
    end of the simulation unperturbed, and could be compared to the all other clones
    generated in the same way. This would make it possible to evaluate the effect
    of each user interaction onto the final result of the simulation, and visualize it
    as a trace instead of a dot in the HA-Plot.

    \item \textbf{Reinforcement trained teacher:} At the moment the simulation
    is consisting of students that behave like an autonomous study group.
    With the teacher interactions described in the previously, one could train a
    virtual teacher using reinforcement learning (RL) with
    the objective to maximize happiness and attention of the class.
    
    \bb

    As there is a fast amount of literature on different teaching methodologies,
    it would be interesting to study if the RL trained teacher applies any of the known
    methodologies or applies new ones. Another interesting aspect would be to study the
    effect different classroom profiles have on the trained teacher, with other
    word, how different classroom profiles form and shape teacher behavior.

    \item \textbf{Screening of classroom profiles:} One obvious extension based
    on the batch processing capabilities of the simulation is to perform a kind
    of screening studying. One would systematically evaluate a high number of personality
    profiles, similar to screening studies in Bioinformatics, comparing thousands
    of combination in order to find interesting tipping points, extremes and curious
    singularities.

    \item \textbf{Dominance hierarchy:} The current simulation when calculating
    the peer pressure on a individual agent is modeling a flat social hierarchy.
    One could extend this to a more realistic Dominance Hierarchy, giving different
    agents different weights in controlling the classroom interest. Such a Dominance
    hierarchy in place one could extend its effect on other actions like chat or
    quarrel, making the outcome of interactions depend as well on the position
    of agents in the dominance hierarchy.

    \item \textbf{Learning Agents:} Because of simplicity the agents have no capability
    to learn and adapt their behavior over time. One could prevent some simple learning
    mechanisms, that would open a wide range of new questions one can study, concerning
    the dynamics of agent and group behavior over longer durations of time.
\end{itemize}

\section{Acknowledgement}
We especially want to thank Prof. Dr. Michael Kickmeier-Rust for his continuos
support and mentoring. Without him this work would have not been possible.
